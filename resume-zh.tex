%%
%% Copyright (c) 2018-2019 Weitian LI <wt@liwt.net>
%% CC BY 4.0 License
%%
%% Created: 2018-04-11
%%

% Chinese version
\documentclass[zh]{resume}

% Adjust icon size (default: same size as the text)
\iconsize{\Large}

% File information shown at the footer of the last page
\fileinfo{%
  \faCopyright{} 2021-2022, Xin 'GinShio' Liu \hspace{0.5em}
  \creativecommons{by-nc-sa}{4.0} \hspace{0.5em}
  \githublink{GinShio}{resume} \hspace{0.5em}
  \faEdit{} \today
}

\name{鑫}{刘}

\keywords{Linux, Programming, Python, C++, Shell}

% \tagline{\icon{\faBinoculars}} <position-to-look-for>}
% \tagline{<current-position>}

% \photo{<height>}{<filename>}

\profile{
  \mobile{176 9119 3046}
  \email{ginshio78@gmail.com}
  \iconlink{\faBlog}{https://blog.ginshio.org}{Blog}\\
  \university{西安科技大学}
  \birthday{1999-06-26}
  \address{陕西·西安}
  \github{GinShio}
  % Custom information:
  % \icontext{<icon>}{<text>}
  % \iconlink{<icon>}{<link>}{<text>}
}

\begin{document}
\makeheader

%======================================================================
% Summary & Objectives
%======================================================================
{\onehalfspacing\hspace{2em}%
计算机科学与技术专业本科生,热衷计算机相关技术,有 3 年 Linux 使用经验,了解 C++ 、elixir 编程,目前正在积极学习 \texttt{C++} 、\texttt{erlang} 以及 \texttt{UNP} 相关知识。
\par}

% ======================================================================
\sectionTitle{计算机技能}{\faWrench}
% ======================================================================
\begin{competences}
  \comptence{操作系统}{%
    OpenSUSE (3 年)%
  }
  \comptence{编程}{%
    C++ / Python / GoLang / Elixir
  }
  \comptence{工具}{%
    Git / Emacs / GPG / \LaTeX
  }
  \comptence{网站开发}{%
    Gin Framework, Phoenix Framework
  }
\end{competences}

\begin{itemize}
  \item {使用 VPS 部署个人域名邮箱、博客、Git、私有网盘等服务}
  \item {使用 DOMjudge 搭建校程序设计竞赛平台}
\end{itemize}

% ======================================================================
\sectionTitle{教育背景}{\faGraduationCap}
% ======================================================================
\begin{experiences}
  \experience%
  [2017.09]%
  {2022.06}%
  {\textbf{西安科技大学} • 计算机科学与技术学院}%
  [\begin{itemize}
     \item {社团活动:校软件实验室副班⻓}
     \item {竞赛经历:蓝桥杯省二,全国高校程序设计大赛天梯赛省三}
   \end{itemize}]
\end{experiences}

% ======================================================================
\sectionTitle{个人项目}{\faCode}
% ======================================================================
\begin{experiences}
  \experience%
  [2021.03]%
  {2021.06}%
  {\textbf{AstraeaOJ}}%
  [作为毕业项目设计的一个 oj 项目,web 后端部分是由 elixir 以及 phoenix framework 实现的 rest http 客户端,代码运行计算部分采用 c++ 实现的建议沙盒部分。\\该项目另外使用了 apache thrift 作为序列化工具,方便 elixir、c++ 两个语言之间的交换数据时所需的结构序列化。代码运行计算部分与 web 后端之间采用 rabbitmq 进行数据收发。]

  \separator{1.2ex}

  \experience%
  [2020.05]%
  {2020.06}%
  {\href{https://github.com/ginshio/stl}{\bf STL 容器}}%
  [遵循c++11标准的 stl实现,努力将stl的六大组件一一实现,当前首先考虑 容器以及容器适配器的实现,并对其进行一定的扩展。\\该项目主要通过实现stl来学习c++方面的知识,尤其是异常、模板以及traits,其次学习其中所包含的算法、数据结构以及设计思想。目前实现了stl容器部分的 顺序容器、有序关联容器和适配器。]

  \separator{1.2ex}

  \experience%
  [2020.02]%
  {2020.04}%
  {\bf GPS 定位测量管理平台}%
  [gps定位测量管理平台,以网站的形式,对用戶使用的设备、测量生成的数据,进行统一的导出、管理、展示,使管理员和用戶都能够清晰的了解设备的使用情况。\\项目采用前后端分离的方式,由我负责整个后端项目的开发工作,后端主要由与用戶交互、数据管理的web端和与设备交互的tcp端组成。web端只要采用 golang 的 gin 框架,提供 restful ⻛格 api 供前端使用,并处理用戶所发出的请求;tcp端使用 gev 框架进行对tcp数据流的解析,分析处理由设备发出的测量结果与其他配置信息请求。]

  \separator{1.2ex}

  \experience%
  [2019.01]%
  {2019.03}%
  {\textbf{校园人脸识别签到系统}}%
  [为校园签到提供人脸信息识别服务,对比一般人脸识别来说,更符合校园签到的管理。前端开发使用 Bootstrap 框架;后端框架采用 Django 框架,并使用 百度paddlepaddle 人工智能框架,使用 Sqlite 进行数据存储服务。本项目荣获 2019年全国大学生服务外包创新创业大赛西北赛区三等奖。]
\end{experiences}

\end{document}
